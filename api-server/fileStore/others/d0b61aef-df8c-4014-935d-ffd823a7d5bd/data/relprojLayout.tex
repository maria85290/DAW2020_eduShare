\documentclass{report}
\usepackage[portuges]{babel}
\usepackage[latin1]{inputenc}

\usepackage{url}
\usepackage{alltt}
\usepackage{listings}
\usepackage{fancyvrb}
\usepackage{graphicx}
\usepackage{algorithmic}
\usepackage[lined,algonl,boxed]{algorithm2e}

\parindent=0pt
\parskip=2pt

\title{Programa��o Imperativa (?�ano de Curso)\\ \textbf{Trabalho Pr�tico N}\\ Relat�rio de Desenvolvimento}
\author{Nome-Aluno1 (numero) \and Nome-Aluno2 (numero) }
\date{\today}

\begin{document}

\maketitle

\begin{abstract}
Isto � um resumo do relat�rio focando o contexto do trb (muito sucinto),
os objectivos concretos e os resultados atingidos.\\
Algum texto curto mas que entusiasme � leitura do relat�rio.
\end{abstract}

\tableofcontents

\chapter{Introdu��o}
Enquadramento.\\
Conte�do do documento.\\
Resultados -- pontos a evidenciar.\\
Estrutura do documento.

\chapter{An�lise e Especifica��o}
\section{Descri��o informal do problema}
\section{Especifica��o do Requisitos}
\subsection{Dados}
\subsection{Pedidos}
\subsection{Rela��es}

\chapter{Concep��o/desenho da Resolu��o}
\section{Estruturas de Dados}
\section{Algoritmos}

\chapter{Codifica��o e Testes}
\section{Alternativas, Decis�es e Problemas de Implementa��o}
\section{Testes realizados e Resultados}
Mostram-se a seguir alguns testes feitos (valores introduzidos) e
os respectivos resultados obtidos:

%\VerbatimInput{teste1.txt}


\chapter{Conclus�o}
S�ntese do Documento.\\
Estado final do projecto; An�lise cr�tica dos resultados.\\
Trabalho futuro.

%\appendix
%\chapter{C�digo do Programa}
%Lista-se a seguir o c�digo \textsf{C} do programa que foi desenvolvido.
%\VerbatimInput{ex1.c}

\end{document}
